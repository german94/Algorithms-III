\documentclass[a4paper]{article}
\usepackage{algorithmicx}
\usepackage{algpseudocode}
\usepackage{graphicx}
\usepackage{vmargin}
\usepackage{mdwlist}
\setpapersize{A4}
\setmargins{2.5cm}       % margen izquierdo
{1.5cm}                        % margen superior
{16.5cm}                      % anchura del texto
{23.42cm}                    % altura del texto
{10pt}                           % altura de los encabezados
{1cm}                           % espacio entre el texto y los encabezados
{0pt}                             % altura del pie de p\'agina
{2cm}  
\begin{document}
\subsection{Resoluci\'on}
Para la resoluci\'on se cuenta inicialmente con una cantidad de v\'ertices n, aristas m, particiones k, y los pesos entre los pares de v\'ertices. Para determinar la k-partici\'on de peso m\'inimo se empezara utilizando un solo conjunto, $conjunto\_1$. Se realizaran todas las combinaciones que existen para esa cantidad de conjuntos, la cual es solamente una (todos los v\'ertices juntos). Y nos guardaremos el peso de esta combinaci\'on en una variable, $suma\_solucion$. Si esta suma es cero, dado que los pesos son mayores o iguales a cero, entonces es la m\'inima que se puede obtener. Esta combinaci\'on es una soluci\'on \'optima. \newline Sino creamos un nuevo conjunto y procedemos a buscar si existe una combinaci\'on, con esta cantidad, cuyo peso sea menor a $suma\_solucion$. Si existe, se actualiza esta variable y esta combinaci\'on reemplazara a la que se tenia como posible soluci\' on. En caso de no existir, se agrega un nuevo conjunto y se proceden a formar nuevas combinaciones. Esto se realizara hasta que exista alguna combinaci\'on donde la suma de todos los conjuntos sea cero. O cuando se terminaron de realizar las combinaciones para k conjuntos. Ya que entonces, habremos visto todas las combinaciones posibles para el m\'aximo de conjuntos que disponemos. Obteniendo una de las posibles soluciones \'optimas.\newline
Para lograr las combinaciones con mas de un conjunto. Se ubicar\'a al v\'ertice 1 en el \'ultimo conjunto que haya. Luego, se procede a ubicar al siguiente v\'ertice en el primero, si es que no exceda a $suma\_solucion$. Una vez ubicado continuamos con el siguiente v\'ertice, siguiendo la misma l\'ogica. Si no es posible ubicar a alguno en un conjunto, se prueba meti\'endolo en el siguiente. En el caso de que un v\'ertice t, 1 $<$ t $<=$ n, no pueda ser ubicado en ninguno de los conjuntos disponibles hasta el momento. Significa que la combinaci\'on que se tiene con los v\'ertices 1 a t-1 no es \'util para avanzar. Por lo que se procede  a remover a t-1 al pr\'oximo conjunto posible (desde la posici\'on que ocupa t-1 actualmente). Si se lo ubic\'o, procedemos a tratar de ubicar nuevamente  a t (empezando desde el primer conjunto) y sino, volvemos a retroceder. Pudiendo caer en dos casos: \newline
Se ubicaron a los n v\'ertices, entonces encontr\'e una combinaci\'on mejor que la que tenia representada en $solucion$, nos guardamos este nuevo peso y la combinaci\'on. Removemos el ultimo v\'ertice para \'ubicarlo en el siguiente conjunto posible. Para asi continuar realizando nuevas combinaciones.\newline
Caso contrario, retroced\'i hasta llegar al v\'ertice 1. Esto significa que termine de observar todas las combinaciones posibles para los conjuntos disponibles actualmente. Por lo que voy a agregar un nuevo conjunto, ubicar al v\'ertice 1 en este ultimo, y proceder con las combinaciones. 
\vspace{0.4cm}
\begin{algorithmic}[1]
\Procedure{ubicar\_vertice}{}
        \State $conjunto\_1 \gets \textit{agregar todos los v\'ertices}$        
        \State $suma\_solucion \gets \textit{sumar el peso del conjunto\_1}$
        \State $solucion \gets conjunto\_1$
        \State $Nuevo\_conjunto \gets \textit{crear un nuevo conjunto }$ 
        \State $k\_particion \gets \textit {agregar el Nuevo\_conjunto}$        
        \State $vertice\_actual \gets 1$
        \For{ i= 1,...,$total$ $de$ $particiones$}
                \If{$(suma\_solucion == 0)$}
                        \State \textit{Return $solucion$}
                \EndIf          
                \State $Nuevo\_conjunto \gets \textit{crear un nuevo conjunto y agregar el vertice\_actual}$
                \State $k\_particion \gets \textit{agregar el Nuevo\_conjunto}$         
                \State $ubicar\_siguientes\_vertices$ $(....i, vertice\_actual_{++},solucion,k\_particion....)$
                \State $\textit{Sacar vertice\_actual del Nuevo\_conjunto}$
        \EndFor 
        \State Return solucion
\EndProcedure
\end{algorithmic}
\newpage
\vspace{0.4cm}
\begin{algorithmic}[1]
\Procedure{ubicar\_siguientes\_vertices}{$..,conjuntos\_disponibles,    vertice\_actual, solucion, k\_particion..$}
        \For {i = 1...$total\_conjuntos$}
                \If {$(suma\_solucion$ $==$ 0)}
                \State $return$ $solucion$
                \EndIf
                \If {\textit{(agregar vertice\_actual al conjunto\_i no excede suma\_solucion)}}
                        \State agrego el vertice\_actual        al conjunto\_i 
                        \If {$(\textit{agregue el \'ultimo v\'ertice)}$}                
                                \State $suma\_solucion \gets \textit{suma de los pesos de cada conjunto en k\_particion}$
                                \State $solucion \gets k\_particion$
                        \Else
                                \State $ubicar\_siguientes\_vertices$ $(....conjuntos\_disponibles, vertice\_actual_{++},solucion,k\_particion....)$
                        \EndIf   
                        \State sacar vertice\_actual del conjunto\_i 
                \EndIf
         \EndFor        
\EndProcedure
\end{algorithmic}
\subsection{Complejidad}
$k\_particion$ es un conjunto de conjuntos de v\'ertices (en �l se van realizando las combinaciones). El mismo se representa con un vector de vectores de enteros. Pero, para un r\'apido acceso a la suma total de los pesos de los conjuntos. Se tendr\'a una tupla, la primer componente es la suma y la segunda el vector.
Para calcular la complejidad total vamos a analizar por partes nuestro algoritmo.\newline
A) Complejidad de la funci\'on $ubicar\_vertice$:\newline
1) Calcular el valor inicial de $suma\_solucion$ (que va a ser el peso de tener a todas los v\'ertices en un solo conjunto). Es decir, sumar m aristas. O(m)\newline
2) Ubicar a todos los v\'ertices en un \'unico posici\'on conjunto que va a ser $solucion$. Se realizan n $push\_back$ en $solucion[0]$. O(n).\newline 
3) Se crea un primer vector $conjunto$ y se lo agrega atr\'as en la segunda componente de $k\_particion$  O(1).\newline
4) Se proceden a realizar las combinaciones, para esto se ejecuta un for que va desde 1 hasta k. En \'el, se van a realizar todas las combinaciones para los i-esimos camiones, 1 $<=$ i $<=k$. Y se van a ir agregando los nuevos conjuntos. En cada iteraci\'on se crea uno, se agrega al v\'ertice 1 en el mismo y se ubica al conjunto atr\'as, en el vector de  $k\_particion$, costo total O(1)(por iteraci\'on). Luego, se procede a llamar a la funci\'on $ubicar\_siguientes\_vertices$ (mas adelante se analiza la complejidad de esta funci\'on). Y por \'ultimo se retira al v\'ertice, pop\_back.
 \newline \newline
B) Complejidad de $ubicar\_siguientes\_vertices$:\newline
Se trata de una funci\'on recursiva, primero vamos a comenzar por analizar que se hace en cada llamada y luego cuantas se hacen en total.\newline
En cada llamada se trata ubicar a un v\'ertice, entre los conjuntos disponibles actualmente. Para esto se realiza un for que va desde 1 hasta el valor de $conjuntos\_disponibles$. \newline
1)En cada iteraci\'on, al comenzar se eval\'ua si el valor de $suma\_solucion$ es cero, O(1). Si no, se procede a  verificar, y de ser posible, a agregar el $vertice\_actual$ al i-esimo conjunto, 1 $<=$ i $<=$ $conjuntos\_disponibles$. Para esto, se le suma a la primer componente de $k\_particion$ el peso que se adiciona el agregar el $vertice\_actual$ a dicho conjunto. Todos los v\'ertices van a estar distribuidos entre los actuales $conjuntos\_disponibles$. Si estoy tratando de ubicar al v\'ertice t, 1 $<$ t $<=$ n,  entonces voy a tener t-1 v\'ertices distribuidos. Pero por cada iteraci\'on solo realizo tantas sumas como v\'ertices haya en el i-esimo conjunto. Es decir que al realizar las $conjuntos\_disponibles$ iteraciones estoy realizando la suma de aristas para t-1 v\'ertices. En el caso de que estos t-1 v\'ertices formen un subgrafo completo tengo (t-1)*(t-2)/2 aristas, que fueron sumadas. \newline
2) Si no se pudo agregar el v\'ertice, se procede a realizar una nueva iteraci\'on. En caso contrario agrego al $vertice\_actual$ atr\'as, en el conjunto correspondiente, push\_back. Si agregue el v\'ertice n, el \'ultimo, se actualiza el valor de $solucion\_suma$ a la de la primer componente de $k\_particion$, O(1). Pero, adem\'as se copia el vector de la segunda componente a $solucion$. Este vector consta de tantas posiciones como $conjuntos\_disponibles$ haya hasta entonces. Con n v\'ertices distribuidos en total. Costo de la copia O(n). \newline En caso de no serlo se realiza una nueva llamada recursiva de la funci\'on pero ahora para ubicar al $vertice\_actual_{++}$. Finalizada la llamada se retira al $vertice\_actual$ del i-esimo conjunto, pop\_back, y se realiza otra iteraci\'on.
Como se detall\'o a lo sumo en cada llamada los costos son O((t-1)*(t-2)/2 + n ), como t esta entre 1 $<=$ t $<=n$ acotamos por n, obteniendo O($n^{2} +n$) = O($n^{2}$).\newline
3) Por \'ultimo calculamos cuantas veces realizamos esta tarea. Cada llamada recursiva va a ubicar a un \'unico v\'ertice. Cada vez que se agrega un nuevo conjunto. Se eval\'ua ubicar nuevamente a los n v\'ertices. Y se vuelven a ejecutar los for's de las funciones recursivas pero, ahora hasta una iteraci\'on m\'as que la llamada anterior.  Como se comienza ubicando al primer v\'ertice en el \'ultimo conjunto creado, y al siguiente desde el primero de los disponibles. Cada vez que un v\'ertice es ubicado en un nuevo conjunto posibilita a lo sumo k conjuntos para el siguiente. Pero lo mismo ocurre con el siguiente a este, as\'i hasta tener los n v\'ertices. En conclusi\'on tenemos las siguientes combinaciones:   
\[
\sum_{V_n=1}^{k}\sum_{V_{n-1}=1}^{k}...... \sum_{V_1=1}^{k}1 = k^{n}
\]
\textit{ Donde V\_s, 1$<=$ s $<=$ n, representa al v\'ertice n\'umero s.}\newline
Y  cada una asociada a los costos mencionados en los \'item 1-2 de la secci\'on B.\newline
El costo total, de ambas secciones seria:
O($m$) + O($n^{2}$)*O($k^{n}$) = O($m$) + O($n^{2}*k^{n}$)\newline 
acotando m por $n*(n-1)/2$, que es el n\'umero m\'aximo de aristas que puede haber para n v\'ertices, tenemos:\newline
O($n^{2}$) + O($n^{2}*k^{n}$) = O($n^{2}*k^{n}$)  
\end{document}