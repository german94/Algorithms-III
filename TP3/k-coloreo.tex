\documentclass[a4paper]{article}
\usepackage{algorithmicx}
\usepackage{algpseudocode}
\usepackage{graphicx}
\usepackage{vmargin}
\usepackage[utf8]{inputenc}
\usepackage{mdwlist}
\setpapersize{A4}
\setmargins{2.5cm}       % margen izquierdo
{1.5cm}                        % margen superior
{16.5cm}                      % anchura del texto
{23.42cm}                    % altura del texto
{10pt}                           % altura de los encabezados
{1cm}                           % espacio entre el texto y los encabezados
{0pt}                             % altura del pie de página
{2cm}                           % espacio entre el texto y el pie de página
\makeatletter
\setlength{\@fptop}{0pt}
\makeatother
\begin{document}

\section{Ejercicio 1}
\subsection{Relación entre k-pmp y el k-coloreo}
Podemos empezar preguntándonos si podemos resolver el problema de $k-coloreo$ con $k-pmp$. La respuesta es si, obteniendo ciertos resultados con $k-pmp$.\newline
Primero hacemos una modificación en el grafo de entrada, poniendo peso 1 a todas las aristas. El problema de   $k-pmp$ busca agrupar los vértices en $k$ conjuntos tales que la suma de los costos(todas las aristas que se generan dentro) en cada conjunto,costo de la partición, sea mínima. Con esta modificación vamos a obtener la mínima cantidad de aristas (porque cada una suma 1) que se generan con k conjuntos.\newline
Sabemos que cada vértice de un conjunto es no adyacente a otro, de otro conjunto, entonces podemos interpretar a cada conjunto como un color. Si el costo de la partición es 0, para tener un coloreo válido necesitamos $k'$ colores, donde $k'$ es la cantidad de conjuntos no vacíos y $k'$ $\leq$ $k$ . \newline
Si todos los conjuntos son no vacíos y el costo de la partición, es mayor que 1 quiere decir que hay por lo menos una arista dentro de un conjunto. Si sucede esto no tenemos un coloreo válido con k colores, porque existen 2 nodos que son adyacentes que tienen el mismo color.
Este es un problema para coloreo pero no para k-pmp, entonces podemos decir que necesitamos 1 color más por cada arista que se genera, la cantidad de aristas es el costo de la partición. Si k-pmp da un costo t, podemos hacer un (k+t)-coloreo y este será válido. \newline
Esto parece no servir mucho para encontrar $\chi$ ($G$), si tenemos un grafo $G$ y $\chi$ ($G$) es $c$, corremos $k-pmp$ con $k$ muy alejado de $c$, vamos a obtener un coloreo válido pero muy alejado del óptimo.\newline
Si bien no encontramos el óptimo corriendo $k-pmp$ una vez podemos, tal vez podemos encontrarlo de alguna forma. Vimos que cuando aumentamos el k se aleja del óptimo, pero ¿Qué pasa si vamos bajando el k? \newline
Corremos $k-pmp$ para $k=n-1$ (con $k=n$ es trivialmente $n-coloreable$) y nos guardamos el costo total $T$. Luego corremos i-pmp para $i$ $\leq$ $n-2$, actualizando $T$ cuando es más grande que el costo obtenido en $i-pmp$, por el costo de $i-pmp$. En algún momento el costo del $i-pmp$ va a ser mayor a $T$ (intuitivamente cuando disminuye $k$, menos puedo separar los nodos sin generar aristas en los conjuntos). \newline %Esto se puede afirmar porque por ejemplo $1-pmp$ es el más costoso, y el costo $2-pmp$ es menor que 1-pmp. \newline
Entonces cuando encontremos un $i-pmp$ cuyo costo sea mayor al $T$, vamos a saber que todos los $T'$ $>$ $T$ van a ser coloreos válidos y los $T''$ $<$ $T$ van a ser no validos con lo cual $\chi$($G$) = $T$ que es el óptimo.\newline
Si bien encontramos el óptimo, la complejidad de encontrarlo de esta forma no es buena. Tenemos que aplicar un algoritmo $k-pmp$ que no es polinomial sino exponencial, en peor caso $n$ veces(es una cota muy grosera). Para un n suficientemente grande el tiempo de ejecución puede ser muy grande, con lo cual para la práctica no sirve este algoritmo.\newline
Ya vimos que se puede resolver k-coloreo con k-pmp y que se puede encontrar el coloreo óptimo.\newline

Ahora vamos a ver que sucede si tenemos $k-coloreo$ y queremos resolver $k-pmp$.\newline
Si un grafo es $k-coloreable$ entonces el resultado de aplicar $k-pmp$ al grafo es costo 0 y $k$ conjuntos sin aristas, sin conjuntos vacíos o $k'$ $<$ $k$ conjuntos sin arista, y el resto conjuntos vacíos. Veamos más detalladamente esto:\newline
Si es $k-coloreable$ tenemos $k$ grupos de vértices con un color cada grupo y en cada grupo los vértices no son adyacentes. Relacionándolo con $k-pmp$ vemos que es casi lo mismo, el costo de la partición va a ser 0, porque siempre busca el mínimo y no hay costo negativos porque todos los costos de las aristas son positivos, vamos a obtener $k$ conjuntos sin aristas o $k'$ $<$ $k$ conjuntos no vacíos.\newline
Si obtengo $k'$ conjuntos no vacíos y costo 0, quiere decir que es $k'-coloreable$. Esto es porque tengo una partición de $k'$ colores cuyo costo es 0, lo cual quiere decir que no tengo aristas en un conjunto, lo cual implica que todo vértice en un conjunto es no adyacente.\newline

\end{document}

\end{document}